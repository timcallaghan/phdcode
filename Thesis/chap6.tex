% chap6.tex (Chapter 6 of the thesis...the conclusion)

\chapter[CONCLUSION AND CLOSING REMARKS]{CONCLUSION AND CLOSING REMARKS}
\label{chap:6}
\section{Discussion and Application to Meteorology}
In this thesis we have presented a detailed picture of how the effects of nonlinearity influence the relationship existing between wavespeed and amplitude for progressive Rossby waves. The techniques utilised have uncovered feature-rich dynamical properties of solutions of the shallow 
atmosphere equations on a rotating sphere. In particular it was shown that \index{resonance!nonlinear}nonlinear resonance plays an important and dominant role for waves with large amplitudes. The effect of resonance was observed to separate solutions of the system into disjoint regions, with similar solutions lying on the same solution branch in wavespeed--amplitude space.

Two specific models of the atmosphere were used: an incompressible and a compressible model. For the incompressible dynamics it was shown that, for slowly propagating progressive waves with longitudinal wavenumber $\kappa=4$, if the amplitude \index{forcing!amplitude}forcing became large enough it was possible for the flow to develop localised low-pressure cells in the mid-latitude regions. These types of extreme amplitude solutions were accompanied by stagnation points in the flow field at locations other than the poles. In general it was observed that for these highly nonlinear waveforms the lower polar free-surface heights, and hence pressures, and also the higher equatorial free-surface elevations, tended to be grossly distorted so that it was common for contours originating near the equator or pole to be deformed towards regions well in excess of the mid-latitudes.

In the Earth's atmosphere, \index{Rossby wave!observational method}Rossby waves are frequently observed by filtering out localised small scale effects and concentrating on the 500 mb height field contours, so that the effects of the planetary boundary layer, and hence friction, are minimal. Using this technique it is possible to expose the large-scale structure of the atmosphere. Since the late 1970's there has been considerable effort expended by meteorologists in an attempt to explain the large-scale process of \index{atmospheric blocking}atmospheric blocking (see the critical review article by Lindzen~\cite{Lindzen:SPW}). In its simplest form, blocking is either the existence of a stationary high pressure cell that persists in the mid-latitudes where westerly flow would normally be observed, or a ridge of high pressure that extends polewards from the tropics and influences the mid-latitude flow characteristics. Usually the formation of a block tends to be associated with stationary planetary waves.  

In the investigation conducted in this thesis, it was found that the most extreme case of large amplitude waves was for slowly moving progressive Rossby waves with the parameters $\kappa=4$ and $\omega=1.0$. The linearized wavespeed associated with this specific parameter configuration was $c\approx 0.395$. Thus for only a slightly smaller value of the zonal flow parameter $\omega$, the linearized and associated nonlinear wavespeeds would be very close to zero, so that the Rossby wave would be approximately \index{Rossby wave!stationary}stationary with respect to the the surface of the Earth. We conjecture that the particular nature of the wavespeed--amplitude relationship for stationary Rossby waves would not be dissimilar to that computed for the case $\kappa=4$ and $\omega=1.0$. If this is so it would imply the existence of highly distorted nearly stationary Rossby waves containing high pressure ridges extending polewards from the equator, with cut-off low pressure cells in the the mid-latitudes. This type of atmospheric configuration could be seen as being crucial to the instigation of a \index{atmospheric blocking}blocking event, with subsequent development of cut-off high-pressure cells near the mid-latitudes when full time dependence is included in the model. 

This argument is, in general, supported by the work of Austin~\cite{Austin:BML} who found that the splitting of westerly winds by blocking is attributable to interference, or resonance, between planetary waves with very large amplitudes. In this conjecture, however, nothing is implied as to how the dynamical system moves from one solution branch to the next. Nor is it expected that all the solution branches would be physically stable. Charney \& DeVore~\cite{Charney:MFE} and Charney, Shukla \& Mo~\cite{Charney:CBB} have conducted studies in an attempt to ascertain the effect of topography and thermal forcing on blocking in the atmosphere and have shown the existence of multiple equilibrium states, some of which exhibit blocking phenomena. It is thus possible that through various \index{forcing!thermal}forcing mechanisms, such as the effects of \index{forcing!topographical}topography and thermal heating/cooling, the atmosphere can pass from one equilibrium state to another. If this proves to be true then the highly nonlinear feature-rich solutions calculated in this work would support the idea that blocking is primarily a dynamical state which is accessible through appropriate forcing of the atmospheric system.

\section{Future work and Closing Remarks}
The aim of this thesis was to investigate numerically nonlinear progressive Rossby wave behaviour. The analysis techniques utilised have uncovered some of the defining features of shallow atmosphere free-surface flow with progressive Rossby waves; however, many questions still remain unanswered. Because of the complexity of the governing equations, and the associated sensitivity of Newton's method, it is highly possible that our numerical solution method did not find all possible solution branches for the particular parameter configurations used. It would thus be desirable to conduct analytical research in an attempt to support the numerical findings contained in this thesis. Additionally, an analytical study would provide insight into the proposed existence of additional solution branches that were conjectured at various stages throughout this thesis. However, because of the sheer complexity of the dynamical equations involved, it is more than likely that such an analytical approach would be severely limited in scope unless significant approximations and idealizations were made.

In this work nothing has been said of the stability of the solutions computed. It seems highly likely that solutions along the first branch of each wavespeed--amplitude curve would be stable, and for small amplitude waves an analysis of the linearized system would more than likely suffice to ascertain the Lyapunov stability of the system. The reasoning behind this statement stems from the fact that Rossby waves are frequently observed in the atmosphere and so must form an integral part of its stable composition. The stability of the highly nonlinear waves computed with very large amplitudes would require special attention again. It would be desirable to know whether or not the largest flow computed for $\kappa=4$ and $\omega=1$, the flow containing cut-off low-pressure cells, is stable, as this would have consequences for the stability of blocking formations, as discussed previously. Additionally, the inclusion of topography and temperature forcing in this analysis would be of great benefit.

This work has shown the sensitive dependence of progressive Rossby-wave solutions upon the zonal flow speed $\omega$. Ideally, a complete study of the entire parameter region $0<\omega<\omega_u$ would reveal the complete dynamical behaviour of large amplitude progressive Rossby waves, where $\omega_u$ is the maximum permissible angular speed, limited by the requirement that the atmosphere have some positive depth. Of course, such a massive numerical study is not achievable. It may be the case that analytical techniques might reveal further dynamical behaviour, although that is beyond the scope of the present work.

